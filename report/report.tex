\documentclass[10pt,a4paper,notitlepage]{report}
\usepackage[utf8]{inputenc}

% For mathematical typesetting.
\usepackage{amsmath}
\usepackage{amsfonts}
\usepackage{amssymb}

% Better rendition of computer modern.
\usepackage{lmodern}

% Provides an extended tabular environment.
\usepackage{tabularx}

% For images.
\usepackage{graphicx}
\usepackage{float}

% For source code listings.
\usepackage{listingsutf8}

% For fancy captions.
\usepackage{caption}

% For hyperlinks and fancy citations.
\usepackage[colorlinks=true,linkcolor=black,citecolor=blue,urlcolor=blue]{hyperref}

\usepackage[left=4cm,right=2cm,top=2cm,bottom=2cm]{geometry}

% Sets spacing between paragraphs
\setlength{\parskip}{0.8em}

% Default caption form
\captionsetup{labelfont=bf,textfont=it,justification=centering,font=footnotesize}

% Shamelessly pinched from 'dbaupp'. cheers!
% http://tex.stackexchange.com/questions/51645/x86-64-assembler-language-dialect-for-the-listings-package
\lstdefinelanguage
    [x64]{Assembler}     % add a "x64" dialect of Assembler
    [x86masm]{Assembler} % based on the "x86masm" dialect
    {   % with these extra keywords:
        morekeywords={
            CDQE,CQO,CMPSQ,CMPXCHG16B,JRCXZ,LODSQ,MOVSXD,
            POPFQ,PUSHFQ,SCASQ,STOSQ,IRETQ,RDTSCP,SWAPGS,
            SYSCALL,
            rax,rdx,rcx,rbx,rsi,rdi,rsp,rbp,
            r8,r8d,r8w,r8b,r9,r9d,r9w,r9b
        }
    }

\lstdefinelanguage
    [3]{Python}
    []{Python}
    {
        deletekeywords={print}
    }

\author{Elliot Thomas\\ \small Supervisor: Sean Tohill\\ \\ University of Westminster}
\title{Network Packet Capture Generation and Falsification}
\date{\today}

\begin{document}
\maketitle
\begin{abstract}
\begin{center}
A document describing the design and development of a tool to assist in the crafting of network packet captures for educational purposes.
\end{center}
\end{abstract}
\pagebreak
\tableofcontents

%%%%%%%%%%%%%%%%
% Introduction %
%%%%%%%%%%%%%%%%
\chapter{Introduction}
Education is important. Education is one of the underpinnings of a progressive society, along with law and order and a few other things.
Despite this, it seems human ingenuity is education's greatest enemy. Humans are lazy - most of us will strive to do as little work as possible for the greatest gain.
We seek to make our lives better, easier, all to spend more time on the things we enjoy. For many, this will be leisure activities, for some, socialising, the humble few may take joy in aulturism.
But motivations are irrelevant here, just the effects.
In academia, plagiarism is problem - people are unwilling to put the time in to do their own work. They realise that if someone else has solved a problem identical to theirs, they could just reuse that.
What these people fail to realise, is that there is no substitute for experience. This problem is relevant in all fields, academic or applied, practical or theoretical; computer security and forensics is no exception.

\section{Reader level}
The reader is expected to be somewhat familiar with networking terminology and forensic analysis techniques.

\section{The problem}
Teaching network security and forensics is aided greatly by the presentation of example packet captures. These are used to learn and practice analytical skills, and to assess the level of understanding and knowledge that students hold.
Creating such packet captures is not difficult, but there are few, if any tools available for automating the process. Changing this is the purpose of this project.

\pagebreak

\section{Existing Solutions}
None, it would seem.

There are a number of tools for network packet capture, and a number of tools for analysing them. These tools do not solve the problem \emph{per se}, but do allow packet capture manipulation to a certain extent, and therefore might be a useful aid in manipulating packet captures.

\subsection{editcap}
Editcap is a program distributed as part of Wireshark.\\
From the manpage\cite{editcap-man}:
\begin{quote}
\textbf{Editcap} is a program that reads some or all of the captured packets from the \underline{infile}, optionally converts them in various ways and writes the resulting packets to the capture \underline{outfile} (or outfiles).
\end{quote}

For the purposes of assisting generation, this program allows splitting a file into a series of files, according to the time they were sent; i.e., splitting every 2 seconds (where each set represents a packet capture and each number represents the time a packet arrived):\\
\indent (1, 1, 2, 3, 5, 7, 8) $\rightarrow$ (1, 1, 2), (3), (5), (7, 8)

\subsection{tshark}
Tshark is a program distributed as part of Wireshark.\\
From the manpage\cite{tshark-man}:
\begin{quote}
\textbf{TShark} is a network protocol analyzer. It lets you capture packet data from a live network, or read packets from a previously saved capture file, either printing a decoded form of those packets to the standard output or writing the packets to a file.  \textbf{TShark}'s native capture file format is \textbf{pcap} format, which is also the format used by \textbf{tcpdump} and various other tools.
\end{quote}

As stated, it acts as a network traffic capture and analysis tool. It has functionality that will identify packets by protocol (and can reconstruct TCP streams) and allows the user to filter them. This is useful functionality, as it allows you to remove packets from certain protocols making captures simpler to understand.

%%%%%%%%%%%%
% Research %
%%%%%%%%%%%%
\chapter{Research}
\section{Approaches to the Problem}
The problem `\emph{How does one generate unique packet captures for teaching analysis to students?}' is open to multiple solutions. In this project, three possible approaches were considered. These can be described as \emph{synthesis}, \emph{generation} and \emph{composition}.

\subsection{Synthesis}
This would work as a program that understands a \emph{scenario} defined in a kind of declarative language, that creates a set of unique permutations of that scenario by varying some key data (names, hosts, times etc.) and the exact sequence of events. From this, it would synthesise a complete packet capture from each scenario - carefully constructing every packet - such that all generated captures represent the same abstract event (i.e. corporate espionage) but with wildly different specifics, hindering collusion.

This approach has the advantage of a flexible interface and fast implementation - at the cost of having to understand and reproduce the complexity of various networking protocols and producing a completely artificial output.

\subsection{Generation}
Much like \emph{synthesis}, this would understand a declarative language for describing a scenario, but this approach would be to program a network of computers to actually do the actions in the scenario rather then synthesise their side effects. This would then run a packet sniffer on this network, generating perfectly authentic network captures of a real network, but where the human actors in a scenario are emulated by computers.

This approach has the advantage of flexibility and authenticity - at the cost of being very inefficient and fragile. It's more then likely such an approach would use a network of virtual machines running real-world operating systems and protocol implementations. This introduces a fair amount of unpredictability (which is both a good and bad thing), and would require changes if/when a used interface changes.

\subsection{Composition}
This was the first approach considered, and is perhaps the simplest. This would be a program that takes a set of existing packet captures (ideally, small purpose-built captures representing single transactions) and combines them while altering things like hosts, times and the like.

This approach has the advantage that real-world scenarios can be used - an existing library of captures will still be useful to provide a data source for the program - and it retains a degree of authenticity. It has a lesser variant of the downside that \emph{synthesis} has, in that it will need to identify and modify various networking protocols, but does not need to understand how they work - which removes a great deal of complexity. It also depends on existing captures - if these aren't available, the tool cannot function.

\subsection{Chosen Approach}
\emph{Synthesis} and \emph{generation} approaches are very complex - a good implementation would take a considerable amount of time to write and test, and allow a lesser degree of control to the user. For this reason, a \emph{composition} based approach was chosen - it is comparatively simple and easy to understand; desirable qualities for any tool.

\section{Programming language}
Choosing a programming language is, obviously, a decision that needs to be made, and there is no single correct choice. There are many factors influencing such a decision, each one needs to be considered. There were three major factors considered in this decision: requirements - what does the program require? knowledge - how difficult will it be to program in and maintain? availability - on what platforms can projects using this language be used?

\subsection{Requirements influence}
At it's heart, this project is a data processing project. There are no requirements for real-time processing, no requirements for concurrency, no requirement to make use of or implement a specific API nor requirements for any kind of interactive behaviour. This project can very easily be designed and implemented as a batch program - give input, run program, get output.

Given the minimalistic requirements of the program, just about any Turing-complete language is serviceable. As such, the decision will have to be made predominantly on the other three factors. That's not to say entirely, some languages lend themselves quite well to generic data manipulation, while others are more specialised and geared towards specific purposes. For instance, Javascript is more suited towards web-based projects (given that it's usual interpreter is a web browser), while C, Java and Python are more general purpose. Haskell, Java and Python have good support for abstract data structures while various assembly languages barely have the notion of a data type.

One important factor is for the program to be easily extended. While it is possible to write an extendible program in just about any language, it helps considerably if there is support for loading code from an external source at runtime, i.e. to allow building a plugin architecture.
Most languages or platforms allow this in some manner (and practically all platforms depend on doing this in one way or another - this is how shared libraries work). While POSIX platforms provide functions like \lstinline$void *dlopen(const char *filename, int flag);$, this interface isn't consistently available on non-POSIX platforms, and requires code to be compiled first. Meanwhile, Python, Javascript and even Bash provide methods for loading new code programatically without the need to compile code first, allowing for a very flexible, powerful and user-friendly system to be developed.

\subsection{Knowledge influence}
Knowledge is a subjective factor that pertains to the programmer(s) developing the project. It is a limiting factor; a programmer proficient in C is not necessarily going to be able to understand a very different language such as Haskell. Also worth mentioning is simple preference - while not a overriding factor - can help shape a choice.

\subsection{Availability influence}
Availability for most languages is somewhat of a non-issue. Given that portability is a desirable outcome, only languages which have a usable and consistent enough implementation across platforms will be considered. This eliminates some languages such as C\#, Visual Basic (or any .NET language).

It is also worth noting portability - are there provisions for file input/output? In the case of the various assembly languages, this is left to the programmer. Even if the assembly language itself is abstract and portable, it only dictates how the processor is used - assembly often leads to some \emph{very} specific code.

\begin{lstlisting}[
	language={[x64]Assembler},captionpos=b,
	caption={[64-bit assembly example]Written for x86-64 processors, this uses Linux's 64-bit system call convention to write data to 'stdout' (standard output)}
	]
mov rax, 1   ; system call constant for write on Linux64
mov rdi, 1   ; first argument: file descriptor to write to, eg: stdout.
mov rsi, msg ; second argument: buffer holding data to be written.
mov rdx, len ; third argument: number of bytes to write.
syscall      ; Do the system call. Similair to int 0x80.
\end{lstlisting}

\subsection{Chosen Language}
Time was the unspoken major factor in this decision, and given the liberty to choose any programming language means ones in which the programmer has experience will take precedence. This means that languages designed for unfamiliar paradigms (such as Haskell, being a purely functional programming language\cite{haskfunc}) or languages that leave a considerable amount of work to their user (such as Assembly) were eliminated.

The choice was Python. This was driven by it's status as a multi-paradigm language that offers a different models of program abstraction.
To illustrate this, below are two functionally equivalent and compatible prime number \emph{iterators};
\begin{lstlisting}[
	language={[3]Python},label={lst:ooprime},
	caption={[Python primegen, object style]An object oriented approach to a prime generator.}
]
class primegen:
    def __init__(self, start, end):
        self.cur = start
        self.lim = end

    def __iter__(self):
        return self

    def __next__(self):
        prime = False
        while not prime:
            if self.cur >= self.lim:
                raise StopIteration
            for i in range(2, self.cur/2):
                if self.cur % i == 0:
                    break
            else:
                retval = self.cur
                prime = True
            self.cur += 1
        return retval
\end{lstlisting}

\begin{lstlisting}[
	language={[3]Python},label={lst:ooimp},
	caption={[Python primegen, imperative style]An imperative approach to a prime generator.}
]
def primegen(cur, lim):
    while cur < lim:
        for i in range(2, cur/2):
            if cur % i == 0:
                break
        else:
            yield cur
        cur += 1

\end{lstlisting}

\subsection{Python 2 or Python 3?}

\section{Libraries}

%%%%%%%%%%%%%%%%
% Requirements %
%%%%%%%%%%%%%%%%
\chapter{Requirements}

%%%%%%%%%%
% Design %
%%%%%%%%%%
\chapter{Design}


\chapter{Appendix}
\section{Formats}
Pcap file format.

\begin{thebibliography}{99}
\bibitem{editcap-man}
    Richard Sharpe, Guy Harris, Ulf Lamping, 2014-11-12\\
    EDITCAP (1), The Wireshark Network Analyzer

\bibitem{tshark-man}
	Gerald Combs, (numerous others)\\
	WIRESHARK (1), The Wireshark Network Analyzer

\bibitem{haskfunc}
    The Haskell 98 Report, Introduction\\
    
    \url{https://www.haskell.org/onlinereport/intro.html} (December, 2014)

\end{thebibliography}
\end{document}
